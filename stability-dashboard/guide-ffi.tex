\documentclass[]{article}
\usepackage{lmodern}
\usepackage{amssymb,amsmath}
\usepackage{ifxetex,ifluatex}
\usepackage{fixltx2e} % provides \textsubscript
\ifnum 0\ifxetex 1\fi\ifluatex 1\fi=0 % if pdftex
  \usepackage[T1]{fontenc}
  \usepackage[utf8]{inputenc}
\else % if luatex or xelatex
  \ifxetex
    \usepackage{mathspec}
    \usepackage{xltxtra,xunicode}
  \else
    \usepackage{fontspec}
  \fi
  \defaultfontfeatures{Mapping=tex-text,Scale=MatchLowercase}
  \newcommand{\euro}{€}
\fi
% use upquote if available, for straight quotes in verbatim environments
\IfFileExists{upquote.sty}{\usepackage{upquote}}{}
% use microtype if available
\IfFileExists{microtype.sty}{\usepackage{microtype}}{}
\usepackage{color}
\usepackage{fancyvrb}
\newcommand{\VerbBar}{|}
\newcommand{\VERB}{\Verb[commandchars=\\\{\}]}
\DefineVerbatimEnvironment{Highlighting}{Verbatim}{commandchars=\\\{\}}
% Add ',fontsize=\small' for more characters per line
\newenvironment{Shaded}{}{}
\newcommand{\KeywordTok}[1]{\textcolor[rgb]{0.00,0.44,0.13}{\textbf{{#1}}}}
\newcommand{\DataTypeTok}[1]{\textcolor[rgb]{0.56,0.13,0.00}{{#1}}}
\newcommand{\DecValTok}[1]{\textcolor[rgb]{0.25,0.63,0.44}{{#1}}}
\newcommand{\BaseNTok}[1]{\textcolor[rgb]{0.25,0.63,0.44}{{#1}}}
\newcommand{\FloatTok}[1]{\textcolor[rgb]{0.25,0.63,0.44}{{#1}}}
\newcommand{\CharTok}[1]{\textcolor[rgb]{0.25,0.44,0.63}{{#1}}}
\newcommand{\StringTok}[1]{\textcolor[rgb]{0.25,0.44,0.63}{{#1}}}
\newcommand{\CommentTok}[1]{\textcolor[rgb]{0.38,0.63,0.69}{\textit{{#1}}}}
\newcommand{\OtherTok}[1]{\textcolor[rgb]{0.00,0.44,0.13}{{#1}}}
\newcommand{\AlertTok}[1]{\textcolor[rgb]{1.00,0.00,0.00}{\textbf{{#1}}}}
\newcommand{\FunctionTok}[1]{\textcolor[rgb]{0.02,0.16,0.49}{{#1}}}
\newcommand{\RegionMarkerTok}[1]{{#1}}
\newcommand{\ErrorTok}[1]{\textcolor[rgb]{1.00,0.00,0.00}{\textbf{{#1}}}}
\newcommand{\NormalTok}[1]{{#1}}
\ifxetex
  \usepackage[setpagesize=false, % page size defined by xetex
              unicode=false, % unicode breaks when used with xetex
              xetex]{hyperref}
\else
  \usepackage[unicode=true]{hyperref}
\fi
\hypersetup{breaklinks=true,
            bookmarks=true,
            pdfauthor={},
            pdftitle={The Rust Foreign Function Interface Guide},
            colorlinks=true,
            citecolor=blue,
            urlcolor=blue,
            linkcolor=magenta,
            pdfborder={0 0 0}}
\urlstyle{same}  % don't use monospace font for urls
\setlength{\parindent}{0pt}
\setlength{\parskip}{6pt plus 2pt minus 1pt}
\setlength{\emergencystretch}{3em}  % prevent overfull lines
\setcounter{secnumdepth}{5}

\title{The Rust Foreign Function Interface Guide}

\begin{document}
\maketitle

0.11.0 (305cf1f1098dc87e346003e6bf7e7cca5705a135 2014-07-10 11:19:20 -0700)

Copyright © 2011-2014 The Rust Project Developers. Licensed under the
\href{http://www.apache.org/licenses/LICENSE-2.0}{Apache License,
Version 2.0} or the \href{http://opensource.org/licenses/MIT}{MIT
license}, at your option.

This file may not be copied, modified, or distributed except according
to those terms.

{
\hypersetup{linkcolor=black}
\setcounter{tocdepth}{3}
\tableofcontents
}
\section{Introduction}\label{introduction}

This guide will use the \href{https://github.com/google/snappy}{snappy}
compression/decompression library as an introduction to writing bindings
for foreign code. Rust is currently unable to call directly into a C++
library, but snappy includes a C interface (documented in
\href{https://github.com/google/snappy/blob/master/snappy-c.h}{\texttt{snappy-c.h}}).

The following is a minimal example of calling a foreign function which
will compile if snappy is installed:

\begin{verbatim}
extern crate libc;
use libc::size_t;

#[link(name = "snappy")]
extern {
    fn snappy_max_compressed_length(source_length: size_t) -> size_t;
}

fn main() {
    let x = unsafe { snappy_max_compressed_length(100) };
    println!("max compressed length of a 100 byte buffer: {}", x);
}
\end{verbatim}

The \texttt{extern} block is a list of function signatures in a foreign
library, in this case with the platform's C ABI. The
\texttt{\#{[}link(...){]}} attribute is used to instruct the linker to
link against the snappy library so the symbols are resolved.

Foreign functions are assumed to be unsafe so calls to them need to be
wrapped with \texttt{unsafe \{\}} as a promise to the compiler that
everything contained within truly is safe. C libraries often expose
interfaces that aren't thread-safe, and almost any function that takes a
pointer argument isn't valid for all possible inputs since the pointer
could be dangling, and raw pointers fall outside of Rust's safe memory
model.

When declaring the argument types to a foreign function, the Rust
compiler can not check if the declaration is correct, so specifying it
correctly is part of keeping the binding correct at runtime.

The \texttt{extern} block can be extended to cover the entire snappy
API:

\begin{verbatim}
extern crate libc;
use libc::{c_int, size_t};

#[link(name = "snappy")]
extern {
    fn snappy_compress(input: *const u8,
                       input_length: size_t,
                       compressed: *mut u8,
                       compressed_length: *mut size_t) -> c_int;
    fn snappy_uncompress(compressed: *const u8,
                         compressed_length: size_t,
                         uncompressed: *mut u8,
                         uncompressed_length: *mut size_t) -> c_int;
    fn snappy_max_compressed_length(source_length: size_t) -> size_t;
    fn snappy_uncompressed_length(compressed: *const u8,
                                  compressed_length: size_t,
                                  result: *mut size_t) -> c_int;
    fn snappy_validate_compressed_buffer(compressed: *const u8,
                                         compressed_length: size_t) -> c_int;
}
# fn main() {}
\end{verbatim}

\section{Creating a safe interface}\label{creating-a-safe-interface}

The raw C API needs to be wrapped to provide memory safety and make use
of higher-level concepts like vectors. A library can choose to expose
only the safe, high-level interface and hide the unsafe internal
details.

Wrapping the functions which expect buffers involves using the
\texttt{slice::raw} module to manipulate Rust vectors as pointers to
memory. Rust's vectors are guaranteed to be a contiguous block of
memory. The length is number of elements currently contained, and the
capacity is the total size in elements of the allocated memory. The
length is less than or equal to the capacity.

\begin{verbatim}
# extern crate libc;
# use libc::{c_int, size_t};
# unsafe fn snappy_validate_compressed_buffer(_: *const u8, _: size_t) -> c_int { 0 }
# fn main() {}
pub fn validate_compressed_buffer(src: &[u8]) -> bool {
    unsafe {
        snappy_validate_compressed_buffer(src.as_ptr(), src.len() as size_t) == 0
    }
}
\end{verbatim}

The \texttt{validate\_compressed\_buffer} wrapper above makes use of an
\texttt{unsafe} block, but it makes the guarantee that calling it is
safe for all inputs by leaving off \texttt{unsafe} from the function
signature.

The \texttt{snappy\_compress} and \texttt{snappy\_uncompress} functions
are more complex, since a buffer has to be allocated to hold the output
too.

The \texttt{snappy\_max\_compressed\_length} function can be used to
allocate a vector with the maximum required capacity to hold the
compressed output. The vector can then be passed to the
\texttt{snappy\_compress} function as an output parameter. An output
parameter is also passed to retrieve the true length after compression
for setting the length.

\begin{verbatim}
# extern crate libc;
# use libc::{size_t, c_int};
# unsafe fn snappy_compress(a: *const u8, b: size_t, c: *mut u8,
#                           d: *mut size_t) -> c_int { 0 }
# unsafe fn snappy_max_compressed_length(a: size_t) -> size_t { a }
# fn main() {}
pub fn compress(src: &[u8]) -> Vec<u8> {
    unsafe {
        let srclen = src.len() as size_t;
        let psrc = src.as_ptr();

        let mut dstlen = snappy_max_compressed_length(srclen);
        let mut dst = Vec::with_capacity(dstlen as uint);
        let pdst = dst.as_mut_ptr();

        snappy_compress(psrc, srclen, pdst, &mut dstlen);
        dst.set_len(dstlen as uint);
        dst
    }
}
\end{verbatim}

Decompression is similar, because snappy stores the uncompressed size as
part of the compression format and \texttt{snappy\_uncompressed\_length}
will retrieve the exact buffer size required.

\begin{verbatim}
# extern crate libc;
# use libc::{size_t, c_int};
# unsafe fn snappy_uncompress(compressed: *const u8,
#                             compressed_length: size_t,
#                             uncompressed: *mut u8,
#                             uncompressed_length: *mut size_t) -> c_int { 0 }
# unsafe fn snappy_uncompressed_length(compressed: *const u8,
#                                      compressed_length: size_t,
#                                      result: *mut size_t) -> c_int { 0 }
# fn main() {}
pub fn uncompress(src: &[u8]) -> Option<Vec<u8>> {
    unsafe {
        let srclen = src.len() as size_t;
        let psrc = src.as_ptr();

        let mut dstlen: size_t = 0;
        snappy_uncompressed_length(psrc, srclen, &mut dstlen);

        let mut dst = Vec::with_capacity(dstlen as uint);
        let pdst = dst.as_mut_ptr();

        if snappy_uncompress(psrc, srclen, pdst, &mut dstlen) == 0 {
            dst.set_len(dstlen as uint);
            Some(dst)
        } else {
            None // SNAPPY_INVALID_INPUT
        }
    }
}
\end{verbatim}

For reference, the examples used here are also available as an
\href{https://github.com/thestinger/rust-snappy}{library on GitHub}.

\section{Stack management}\label{stack-management}

Rust tasks by default run on a ``large stack''. This is actually
implemented as a reserving a large segment of the address space and then
lazily mapping in pages as they are needed. When calling an external C
function, the code is invoked on the same stack as the rust stack. This
means that there is no extra stack-switching mechanism in place because
it is assumed that the large stack for the rust task is plenty for the C
function to have.

A planned future improvement (net yet implemented at the time of this
writing) is to have a guard page at the end of every rust stack. No rust
function will hit this guard page (due to Rust's usage of LLVM's
\texttt{\_\_morestack}). The intention for this unmapped page is to
prevent infinite recursion in C from overflowing onto other rust stacks.
If the guard page is hit, then the process will be terminated with a
message saying that the guard page was hit.

For normal external function usage, this all means that there shouldn't
be any need for any extra effort on a user's perspective. The C stack
naturally interleaves with the rust stack, and it's ``large enough'' for
both to interoperate. If, however, it is determined that a larger stack
is necessary, there are appropriate functions in the task spawning API
to control the size of the stack of the task which is spawned.

\section{Destructors}\label{destructors}

Foreign libraries often hand off ownership of resources to the calling
code. When this occurs, we must use Rust's destructors to provide safety
and guarantee the release of these resources (especially in the case of
failure).

\section{Callbacks from C code to Rust
functions}\label{callbacks-from-c-code-to-rust-functions}

Some external libraries require the usage of callbacks to report back
their current state or intermediate data to the caller. It is possible
to pass functions defined in Rust to an external library. The
requirement for this is that the callback function is marked as
\texttt{extern} with the correct calling convention to make it callable
from C code.

The callback function that can then be sent to through a registration
call to the C library and afterwards be invoked from there.

A basic example is:

Rust code:

\begin{verbatim}
extern fn callback(a:i32) {
    println!("I'm called from C with value {0}", a);
}

#[link(name = "extlib")]
extern {
   fn register_callback(cb: extern fn(i32)) -> i32;
   fn trigger_callback();
}

fn main() {
    unsafe {
        register_callback(callback);
        trigger_callback(); // Triggers the callback
    }
}
\end{verbatim}

C code:

\begin{Shaded}
\begin{Highlighting}[]
\KeywordTok{typedef} \DataTypeTok{void} \NormalTok{(*rust_callback)(}\DataTypeTok{int32_t}\NormalTok{);}
\NormalTok{rust_callback cb;}

\DataTypeTok{int32_t} \NormalTok{register_callback(rust_callback callback) \{}
    \NormalTok{cb = callback;}
    \KeywordTok{return} \DecValTok{1}\NormalTok{;}
\NormalTok{\}}

\DataTypeTok{void} \NormalTok{trigger_callback() \{}
  \NormalTok{cb(}\DecValTok{7}\NormalTok{); }\CommentTok{// Will call callback(7) in Rust}
\NormalTok{\}}
\end{Highlighting}
\end{Shaded}

In this example will Rust's \texttt{main()} will call
\texttt{do\_callback()} in C, which would call back to
\texttt{callback()} in Rust.

\subsection{Targetting callbacks to Rust
objects}\label{targetting-callbacks-to-rust-objects}

The former example showed how a global function can be called from C
code. However it is often desired that the callback is targetted to a
special Rust object. This could be the object that represents the
wrapper for the respective C object.

This can be achieved by passing an unsafe pointer to the object down to
the C library. The C library can then include the pointer to the Rust
object in the notification. This will allow the callback to unsafely
access the referenced Rust object.

Rust code:

\begin{verbatim}

struct RustObject {
    a: i32,
    // other members
}

extern fn callback(target: *mut RustObject, a:i32) {
    println!("I'm called from C with value {0}", a);
    unsafe {
        // Update the value in RustObject with the value received from the callback
        (*target).a = a;
    }
}

#[link(name = "extlib")]
extern {
   fn register_callback(target: *mut RustObject,
                        cb: extern fn(*mut RustObject, i32)) -> i32;
   fn trigger_callback();
}

fn main() {
    // Create the object that will be referenced in the callback
    let mut rust_object = box RustObject { a: 5 };

    unsafe {
        register_callback(&mut *rust_object, callback);
        trigger_callback();
    }
}
\end{verbatim}

C code:

\begin{Shaded}
\begin{Highlighting}[]
\KeywordTok{typedef} \DataTypeTok{void} \NormalTok{(*rust_callback)(}\DataTypeTok{int32_t}\NormalTok{);}
\DataTypeTok{void}\NormalTok{* cb_target;}
\NormalTok{rust_callback cb;}

\DataTypeTok{int32_t} \NormalTok{register_callback(}\DataTypeTok{void}\NormalTok{* callback_target, rust_callback callback) \{}
    \NormalTok{cb_target = callback_target;}
    \NormalTok{cb = callback;}
    \KeywordTok{return} \DecValTok{1}\NormalTok{;}
\NormalTok{\}}

\DataTypeTok{void} \NormalTok{trigger_callback() \{}
  \NormalTok{cb(cb_target, }\DecValTok{7}\NormalTok{); }\CommentTok{// Will call callback(&rustObject, 7) in Rust}
\NormalTok{\}}
\end{Highlighting}
\end{Shaded}

\subsection{Asynchronous callbacks}\label{asynchronous-callbacks}

In the previously given examples the callbacks are invoked as a direct
reaction to a function call to the external C library. The control over
the current thread is switched from Rust to C to Rust for the execution
of the callback, but in the end the callback is executed on the same
thread (and Rust task) that lead called the function which triggered the
callback.

Things get more complicated when the external library spawns its own
threads and invokes callbacks from there. In these cases access to Rust
data structures inside the callbacks is especially unsafe and proper
synchronization mechanisms must be used. Besides classical
synchronization mechanisms like mutexes, one possibility in Rust is to
use channels (in \texttt{std::comm}) to forward data from the C thread
that invoked the callback into a Rust task.

If an asynchronous callback targets a special object in the Rust address
space it is also absolutely necessary that no more callbacks are
performed by the C library after the respective Rust object gets
destroyed. This can be achieved by unregistering the callback in the
object's destructor and designing the library in a way that guarantees
that no callback will be performed after unregistration.

\section{Linking}\label{linking}

The \texttt{link} attribute on \texttt{extern} blocks provides the basic
building block for instructing rustc how it will link to native
libraries. There are two accepted forms of the link attribute today:

\begin{itemize}
\itemsep1pt\parskip0pt\parsep0pt
\item
  \texttt{\#{[}link(name = "foo"){]}}
\item
  \texttt{\#{[}link(name = "foo", kind = "bar"){]}}
\end{itemize}

In both of these cases, \texttt{foo} is the name of the native library
that we're linking to, and in the second case \texttt{bar} is the type
of native library that the compiler is linking to. There are currently
three known types of native libraries:

\begin{itemize}
\itemsep1pt\parskip0pt\parsep0pt
\item
  Dynamic - `\#{[}link(name = ``readline''){]}
\item
  Static - `\#{[}link(name = ``my\_build\_dependency'', kind =
  ``static''){]}
\item
  Frameworks - `\#{[}link(name = ``CoreFoundation'', kind =
  ``framework''){]}
\end{itemize}

Note that frameworks are only available on OSX targets.

The different \texttt{kind} values are meant to differentiate how the
native library participates in linkage. From a linkage perspective, the
rust compiler creates two flavors of artifacts: partial (rlib/staticlib)
and final (dylib/binary). Native dynamic libraries and frameworks are
propagated to the final artifact boundary, while static libraries are
not propagated at all.

A few examples of how this model can be used are:

\begin{itemize}
\itemsep1pt\parskip0pt\parsep0pt
\item
  A native build dependency. Sometimes some C/C++ glue is needed when
  writing some rust code, but distribution of the C/C++ code in a
  library format is just a burden. In this case, the code will be
  archived into \texttt{libfoo.a} and then the rust crate would declare
  a dependency via
  \texttt{\#{[}link(name = "foo", kind =   "static"){]}}.
\end{itemize}

Regardless of the flavor of output for the crate, the native static
library will be included in the output, meaning that distribution of the
native static library is not necessary.

\begin{itemize}
\itemsep1pt\parskip0pt\parsep0pt
\item
  A normal dynamic dependency. Common system libraries (like
  \texttt{readline}) are available on a large number of systems, and
  often a static copy of these libraries cannot be found. When this
  dependency is included in a rust crate, partial targets (like rlibs)
  will not link to the library, but when the rlib is included in a final
  target (like a binary), the native library will be linked in.
\end{itemize}

On OSX, frameworks behave with the same semantics as a dynamic library.

\subsection{The \texttt{link\_args}
attribute}\label{the-linkux5fargs-attribute}

There is one other way to tell rustc how to customize linking, and that
is via the \texttt{link\_args} attribute. This attribute is applied to
\texttt{extern} blocks and specifies raw flags which need to get passed
to the linker when producing an artifact. An example usage would be:

\begin{verbatim}
#![feature(link_args)]

#[link_args = "-foo -bar -baz"]
extern {}
# fn main() {}
\end{verbatim}

Note that this feature is currently hidden behind the
\texttt{feature(link\_args)} gate because this is not a sanctioned way
of performing linking. Right now rustc shells out to the system linker,
so it makes sense to provide extra command line arguments, but this will
not always be the case. In the future rustc may use LLVM directly to
link native libraries in which case \texttt{link\_args} will have no
meaning.

It is highly recommended to \emph{not} use this attribute, and rather
use the more formal \texttt{\#{[}link(...){]}} attribute on
\texttt{extern} blocks instead.

\section{Unsafe blocks}\label{unsafe-blocks}

Some operations, like dereferencing unsafe pointers or calling functions
that have been marked unsafe are only allowed inside unsafe blocks.
Unsafe blocks isolate unsafety and are a promise to the compiler that
the unsafety does not leak out of the block.

Unsafe functions, on the other hand, advertise it to the world. An
unsafe function is written like this:

\begin{verbatim}
unsafe fn kaboom(ptr: *const int) -> int { *ptr }
\end{verbatim}

This function can only be called from an \texttt{unsafe} block or
another \texttt{unsafe} function.

\section{Accessing foreign globals}\label{accessing-foreign-globals}

Foreign APIs often export a global variable which could do something
like track global state. In order to access these variables, you declare
them in \texttt{extern} blocks with the \texttt{static} keyword:

\begin{verbatim}
extern crate libc;

#[link(name = "readline")]
extern {
    static rl_readline_version: libc::c_int;
}

fn main() {
    println!("You have readline version {} installed.",
             rl_readline_version as int);
}
\end{verbatim}

Alternatively, you may need to alter global state provided by a foreign
interface. To do this, statics can be declared with \texttt{mut} so rust
can mutate them.

\begin{verbatim}
extern crate libc;
use std::ptr;

#[link(name = "readline")]
extern {
    static mut rl_prompt: *const libc::c_char;
}

fn main() {
    "[my-awesome-shell] $".with_c_str(|buf| {
        unsafe { rl_prompt = buf; }
        // get a line, process it
        unsafe { rl_prompt = ptr::null(); }
    });
}
\end{verbatim}

\section{Foreign calling conventions}\label{foreign-calling-conventions}

Most foreign code exposes a C ABI, and Rust uses the platform's C
calling convention by default when calling foreign functions. Some
foreign functions, most notably the Windows API, use other calling
conventions. Rust provides a way to tell the compiler which convention
to use:

\begin{verbatim}
extern crate libc;

#[cfg(target_os = "win32", target_arch = "x86")]
#[link(name = "kernel32")]
#[allow(non_snake_case_functions)]
extern "stdcall" {
    fn SetEnvironmentVariableA(n: *const u8, v: *const u8) -> libc::c_int;
}
# fn main() { }
\end{verbatim}

This applies to the entire \texttt{extern} block. The list of supported
ABI constraints are:

\begin{itemize}
\itemsep1pt\parskip0pt\parsep0pt
\item
  \texttt{stdcall}
\item
  \texttt{aapcs}
\item
  \texttt{cdecl}
\item
  \texttt{fastcall}
\item
  \texttt{Rust}
\item
  \texttt{rust-intrinsic}
\item
  \texttt{system}
\item
  \texttt{C}
\item
  \texttt{win64}
\end{itemize}

Most of the abis in this list are self-explanatory, but the
\texttt{system} abi may seem a little odd. This constraint selects
whatever the appropriate ABI is for interoperating with the target's
libraries. For example, on win32 with a x86 architecture, this means
that the abi used would be \texttt{stdcall}. On x86\_64, however,
windows uses the \texttt{C} calling convention, so \texttt{C} would be
used. This means that in our previous example, we could have used
\texttt{extern "system" \{ ... \}} to define a block for all windows
systems, not just x86 ones.

\section{Interoperability with foreign
code}\label{interoperability-with-foreign-code}

Rust guarantees that the layout of a \texttt{struct} is compatible with
the platform's representation in C. A \texttt{\#{[}packed{]}} attribute
is available, which will lay out the struct members without padding.
However, there are currently no guarantees about the layout of an
\texttt{enum}.

Rust's owned and managed boxes use non-nullable pointers as handles
which point to the contained object. However, they should not be
manually created because they are managed by internal allocators.
References can safely be assumed to be non-nullable pointers directly to
the type. However, breaking the borrow checking or mutability rules is
not guaranteed to be safe, so prefer using raw pointers (\texttt{*}) if
that's needed because the compiler can't make as many assumptions about
them.

Vectors and strings share the same basic memory layout, and utilities
are available in the \texttt{vec} and \texttt{str} modules for working
with C APIs. However, strings are not terminated with
\texttt{\textbackslash{}0}. If you need a NUL-terminated string for
interoperability with C, you should use the \texttt{c\_str::to\_c\_str}
function.

The standard library includes type aliases and function definitions for
the C standard library in the \texttt{libc} module, and Rust links
against \texttt{libc} and \texttt{libm} by default.

\section{The ``nullable pointer
optimization''}\label{the-nullable-pointer-optimization}

Certain types are defined to not be \texttt{null}. This includes
references (\texttt{\&T}, \texttt{\&mut T}), owning pointers
(\texttt{\textasciitilde{}T}), and function pointers
(\texttt{extern "abi" fn()}). When interfacing with C, pointers that
might be null are often used. As a special case, a generic \texttt{enum}
that contains exactly two variants, one of which contains no data and
the other containing a single field, is eligible for the ``nullable
pointer optimization''. When such an enum is instantiated with one of
the non-nullable types, it is represented as a single pointer, and the
non-data variant is represented as the null pointer. So
\texttt{Option\textless{}extern "C" fn(c\_int) -\textgreater{} c\_int\textgreater{}}
is how one represents a nullable function pointer using the C ABI.

\end{document}
