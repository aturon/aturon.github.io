\documentclass[]{article}
\usepackage{lmodern}
\usepackage{amssymb,amsmath}
\usepackage{ifxetex,ifluatex}
\usepackage{fixltx2e} % provides \textsubscript
\ifnum 0\ifxetex 1\fi\ifluatex 1\fi=0 % if pdftex
  \usepackage[T1]{fontenc}
  \usepackage[utf8]{inputenc}
\else % if luatex or xelatex
  \ifxetex
    \usepackage{mathspec}
    \usepackage{xltxtra,xunicode}
  \else
    \usepackage{fontspec}
  \fi
  \defaultfontfeatures{Mapping=tex-text,Scale=MatchLowercase}
  \newcommand{\euro}{€}
\fi
% use upquote if available, for straight quotes in verbatim environments
\IfFileExists{upquote.sty}{\usepackage{upquote}}{}
% use microtype if available
\IfFileExists{microtype.sty}{\usepackage{microtype}}{}
\ifxetex
  \usepackage[setpagesize=false, % page size defined by xetex
              unicode=false, % unicode breaks when used with xetex
              xetex]{hyperref}
\else
  \usepackage[unicode=true]{hyperref}
\fi
\hypersetup{breaklinks=true,
            bookmarks=true,
            pdfauthor={},
            pdftitle={Rust Design FAQ},
            colorlinks=true,
            citecolor=blue,
            urlcolor=blue,
            linkcolor=magenta,
            pdfborder={0 0 0}}
\urlstyle{same}  % don't use monospace font for urls
\setlength{\parindent}{0pt}
\setlength{\parskip}{6pt plus 2pt minus 1pt}
\setlength{\emergencystretch}{3em}  % prevent overfull lines
\setcounter{secnumdepth}{5}

\title{Rust Design FAQ}

\begin{document}
\maketitle

0.12.0-pre (2c50add48 2014-07-27 23:03:52 -0700)

Copyright © 2011-2014 The Rust Project Developers. Licensed under the
\href{http://www.apache.org/licenses/LICENSE-2.0}{Apache License,
Version 2.0} or the \href{http://opensource.org/licenses/MIT}{MIT
license}, at your option.

This file may not be copied, modified, or distributed except according
to those terms.

{
\hypersetup{linkcolor=black}
\setcounter{tocdepth}{3}
\tableofcontents
}
This document describes decisions were arrived at after lengthy
discussion and experimenting with alternatives. Please do not propose
reversing them unless you have a new, extremely compelling argument.
Note that this document specifically talks about the \emph{language} and
not any library or implementation.

A few general guidelines define the philosophy:

\begin{itemize}
\itemsep1pt\parskip0pt\parsep0pt
\item
  \href{http://en.wikipedia.org/wiki/Memory_safety}{Memory safety} must
  never be compromised
\item
  \href{http://en.wikipedia.org/wiki/Abstraction_\%28computer_science\%29}{Abstraction}
  should be zero-cost, while still maintaining safety
\item
  Practicality is key
\end{itemize}

\section{Semantics}\label{semantics}

\subsection{Data layout is
unspecified}\label{data-layout-is-unspecified}

In the general case, \texttt{enum} and \texttt{struct} layout is
undefined. This allows the compiler to potentially do optimizations like
re-using padding for the discriminant, compacting variants of nested
enums, reordering fields to remove padding, etc. \texttt{enum}s which
carry no data (``C-like'') are eligible to have a defined
representation. Such \texttt{enum}s are easily distinguished in that
they are simply a list of names that carry no data:

\begin{verbatim}
enum CLike {
    A,
    B = 32,
    C = 34,
    D
}
\end{verbatim}

The
\href{http://doc.rust-lang.org/rust.html\#miscellaneous-attributes}{repr
attribute} can be applied to such \texttt{enum}s to give them the same
representation as a primitive. This allows using Rust \texttt{enum}s in
FFI where C \texttt{enum}s are also used, for most use cases. The
attribute can also be applied to \texttt{struct}s to get the same layout
as a C struct would.

\subsection{There is no GC}\label{there-is-no-gc}

A language that requires a GC is a language that opts into a larger,
more complex runtime than Rust cares for. Rust is usable on bare metal
with no extra runtime. Additionally, garbage collection is frequently a
source of non-deterministic behavior. Rust provides the tools to make
using a GC possible and even pleasant, but it should not be a
requirement for implementing the language.

\subsection{Non-\texttt{Share} \texttt{static mut} is
unsafe}\label{non-share-static-mut-is-unsafe}

Types which are
\href{http://doc.rust-lang.org/core/kinds/trait.Share.html}{\texttt{Share}}
are thread-safe when multiple shared references to them are used
concurrently. Types which are not \texttt{Share} are not thread-safe,
and thus when used in a global require unsafe code to use.

\subsubsection{If mutable static items that implement \texttt{Share} are
safe, why is taking \&mut SHARABLE
unsafe?}\label{if-mutable-static-items-that-implement-share-are-safe-why-is-taking-mut-sharable-unsafe}

Having multiple aliasing \texttt{\&mut T}s is never allowed. Due to the
nature of globals, the borrow checker cannot possibly ensure that a
static obeys the borrowing rules, so taking a mutable reference to a
static is always unsafe.

\subsection{There is no life before or after main (no static
ctors/dtors)}\label{there-is-no-life-before-or-after-main-no-static-ctorsdtors}

Globals can not have a non-constant-expression constructor and cannot
have a destructor at all. This is an opinion of the language. Static
constructors are undesirable because they can slow down program startup.
Life before main is often considered a misfeature, never to be used.
Rust helps this along by just not having the feature.

See
\href{https://mail.mozilla.org/pipermail/rust-dev/2013-April/003815.html}{the
C++ FQA} about the ``static initialization order fiasco'', and
\href{http://ericlippert.com/2013/02/06/static-constructors-part-one/}{Eric
Lippert's blog} for the challenges in C\#, which also has this feature.

A nice replacement is the
\href{https://gist.github.com/Kimundi/8782487}{lazy constructor macro}
by \href{https://github.com/Kimundi}{Marvin Löbel}.

\subsection{The language does not require a
runtime}\label{the-language-does-not-require-a-runtime}

See the above entry on GC. Requiring a runtime limits the utility of the
language, and makes it undeserving of the title ``systems language''.
All Rust code should need to run is a stack.

\subsection{\texttt{match} must be
exhaustive}\label{match-must-be-exhaustive}

\texttt{match} being exhaustive has some useful properties. First, if
every possibility is covered by the \texttt{match}, adding further
variants to the \texttt{enum} in the future will prompt a compilation
failure, rather than runtime failure. Second, it makes cost explicit. In
general, only safe way to have a non-exhaustive match would be to fail
the task if nothing is matched, though it could fall through if the type
of the \texttt{match} expression is \texttt{()}. This sort of hidden
cost and special casing is against the language's philosophy. It's easy
to ignore certain cases by using the \texttt{\_} wildcard:

\texttt{rust,ignore match val.do\_something() \{     Cat(a) =\textgreater{} \{ /* ... */ \}     \_      =\textgreater{} \{ /* ... */ \} \}}

\href{https://github.com/rust-lang/rust/issues/3101}{\#3101} is the
issue that proposed making this the only behavior, with rationale and
discussion.

\subsection{No guaranteed tail-call
optimization}\label{no-guaranteed-tail-call-optimization}

In general, tail-call optimization is not guaranteed: see
\href{https://mail.mozilla.org/pipermail/rust-dev/2013-April/003557.html}{here}
for a detailed explanation with references. There is a
\href{https://github.com/rust-lang/rfcs/pull/81}{proposed extension}
that would allow tail-call elimination in certain contexts. The compiler
is still free to optimize tail-calls
\href{http://llvm.org/docs/CodeGenerator.html\#sibling-call-optimization}{when
it pleases}, however.

\subsection{No constructors}\label{no-constructors}

Functions can serve the same purpose as constructors without adding any
language complexity.

\subsection{No copy constructors}\label{no-copy-constructors}

Types which implement
\href{http://doc.rust-lang.org/core/kinds/trait.Copy.html}{\texttt{Copy}},
will do a standard C-like ``shallow copy'' with no extra work (similar
to ``plain old data'' in C++). It is impossible to implement
\texttt{Copy} types that require custom copy behavior. Instead, in Rust
``copy constructors'' are created by implementing the
\href{http://doc.rust-lang.org/core/clone/trait.Clone.html}{\texttt{Clone}}
trait, and explicitly calling the \texttt{clone} method. Making
user-defined copy operators explicit surfaces the underlying complexity,
forcing the developer to opt-in to potentially expensive operations.

\subsection{No move constructors}\label{no-move-constructors}

Values of all types are moved via \texttt{memcpy}. This makes writing
generic unsafe code much simpler since assignment, passing and returning
are known to never have a side effect like unwinding.

\section{Syntax}\label{syntax}

\subsection{Macros require balanced
delimiters}\label{macros-require-balanced-delimiters}

This is to make the language easier to parse for machines. Since the
body of a macro can contain arbitrary tokens, some restriction is needed
to allow simple non-macro-expanding lexers and parsers. This comes in
the form of requiring that all delimiters be balanced.

\subsection{\texttt{-\textgreater{}} for function return
type}\label{for-function-return-type}

This is to make the language easier to parse for humans, especially in
the face of higher-order functions.
\texttt{fn foo\textless{}T\textgreater{}(f: fn(int): int, fn(T): U): U}
is not particularly easy to read.

\subsection{\texttt{let} is used to introduce
variables}\label{let-is-used-to-introduce-variables}

\texttt{let} not only defines variables, but can do pattern matching.
One can also redeclare immutable variables with \texttt{let}. This is
useful to avoid unnecessary \texttt{mut} annotations. An interesting
historical note is that Rust comes, syntactically, most closely from ML,
which also uses \texttt{let} to introduce bindings.

See also
\href{https://mail.mozilla.org/pipermail/rust-dev/2014-January/008319.html}{a
long thread} on renaming \texttt{let mut} to \texttt{var}.

\end{document}
