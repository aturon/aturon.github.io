\documentclass[]{article}
\usepackage{lmodern}
\usepackage{amssymb,amsmath}
\usepackage{ifxetex,ifluatex}
\usepackage{fixltx2e} % provides \textsubscript
\ifnum 0\ifxetex 1\fi\ifluatex 1\fi=0 % if pdftex
  \usepackage[T1]{fontenc}
  \usepackage[utf8]{inputenc}
\else % if luatex or xelatex
  \ifxetex
    \usepackage{mathspec}
    \usepackage{xltxtra,xunicode}
  \else
    \usepackage{fontspec}
  \fi
  \defaultfontfeatures{Mapping=tex-text,Scale=MatchLowercase}
  \newcommand{\euro}{€}
\fi
% use upquote if available, for straight quotes in verbatim environments
\IfFileExists{upquote.sty}{\usepackage{upquote}}{}
% use microtype if available
\IfFileExists{microtype.sty}{\usepackage{microtype}}{}
\ifxetex
  \usepackage[setpagesize=false, % page size defined by xetex
              unicode=false, % unicode breaks when used with xetex
              xetex]{hyperref}
\else
  \usepackage[unicode=true]{hyperref}
\fi
\hypersetup{breaklinks=true,
            bookmarks=true,
            pdfauthor={},
            pdftitle={Rust Documentation},
            colorlinks=true,
            citecolor=blue,
            urlcolor=blue,
            linkcolor=magenta,
            pdfborder={0 0 0}}
\urlstyle{same}  % don't use monospace font for urls
\setlength{\parindent}{0pt}
\setlength{\parskip}{6pt plus 2pt minus 1pt}
\setlength{\emergencystretch}{3em}  % prevent overfull lines
\setcounter{secnumdepth}{5}

\title{Rust Documentation}

\begin{document}
\maketitle

0.11.0-pre (169c988d09a9d4e46de2b7fead9489e94964c7c7 2014-07-02 18:41:38 +0000)

Copyright © 2011-2014 The Rust Project Developers. Licensed under the
\href{http://www.apache.org/licenses/LICENSE-2.0}{Apache License,
Version 2.0} or the \href{http://opensource.org/licenses/MIT}{MIT
license}, at your option.

This file may not be copied, modified, or distributed except according
to those terms.

{
\hypersetup{linkcolor=black}
\setcounter{tocdepth}{3}
\tableofcontents
}
\texttt{rustdoc} is the built-in tool for generating documentation. It
integrates with the compiler to provide accurate hyperlinking between
usage of types and their documentation. Furthermore, by not using a
separate parser, it will never reject your valid Rust code.

\section{Creating Documentation}\label{creating-documentation}

Documenting Rust APIs is quite simple. To document a given item, we have
``doc comments'':

\begin{verbatim}
# #![allow(unused_attribute)]
// the "link" crate attribute is currently required for rustdoc, but normally
// isn't needed.
#![crate_id = "universe"]
#![crate_type="lib"]

//! Tools for dealing with universes (this is a doc comment, and is shown on
//! the crate index page. The ! makes it apply to the parent of the comment,
//! rather than what follows).

# mod workaround_the_outer_function_rustdoc_inserts {
/// Widgets are very common (this is a doc comment, and will show up on
/// Widget's documentation).
pub struct Widget {
    /// All widgets have a purpose (this is a doc comment, and will show up
    /// the field's documentation).
    purpose: String,
    /// Humans are not allowed to understand some widgets
    understandable: bool
}

pub fn recalibrate() {
    //! Recalibrate a pesky universe (this is also a doc comment, like above,
    //! the documentation will be applied to the *parent* item, so
    //! `recalibrate`).
    /* ... */
}
# }
\end{verbatim}

Documentation can also be controlled via the \texttt{doc} attribute on
items. This is implicitly done by the compiler when using the above form
of doc comments (converting the slash-based comments to
\texttt{\#{[}doc{]}} attributes).

\begin{verbatim}
#[doc = "
Calculates the factorial of a number.

Given the input integer `n`, this function will calculate `n!` and return it.
"]
pub fn factorial(n: int) -> int { if n < 2 {1} else {n * factorial(n)} }
# fn main() {}
\end{verbatim}

The \texttt{doc} attribute can also be used to control how rustdoc emits
documentation in some cases.

\begin{verbatim}
// Rustdoc will inline documentation of a `pub use` into this crate when the
// `pub use` reaches across crates, but this behavior can also be disabled.
#[doc(no_inline)]
pub use std::option::Option;
# fn main() {}
\end{verbatim}

Doc comments are markdown, and are currently parsed with the
\href{https://github.com/vmg/sundown/}{sundown} library. rustdoc does
not yet do any fanciness such as referencing other items inline, like
javadoc's \texttt{@see}. One exception to this is that the first
paragraph will be used as the ``summary'' of an item in the generated
documentation:

\begin{verbatim}
/// A whizbang. Does stuff. (this line is the summary)
///
/// Whizbangs are ...
struct Whizbang;
\end{verbatim}

To generate the docs, run \texttt{rustdoc universe.rs}. By default, it
generates a directory called \texttt{doc}, with the documentation for
\texttt{universe} being in \texttt{doc/universe/index.html}. If you are
using other crates with \texttt{extern crate}, rustdoc will even link to
them when you use their types, as long as their documentation has
already been generated by a previous run of rustdoc, or the crate
advertises that its documentation is hosted at a given URL.

The generated output can be controlled with the \texttt{doc} crate
attribute, which is how the above advertisement works. An example from
the \texttt{libstd} documentation:

\begin{verbatim}
#[doc(html_logo_url = "http://www.rust-lang.org/logos/rust-logo-128x128-blk-v2.png",
      html_favicon_url = "http://www.rust-lang.org/favicon.ico",
      html_root_url = "http://doc.rust-lang.org/")];
\end{verbatim}

The \texttt{html\_root\_url} is the prefix that rustdoc will apply to
any references to that crate's types etc.

rustdoc can also generate JSON, for consumption by other tools, with
\texttt{rustdoc -\/-output-format json}, and also consume
already-generated JSON with \texttt{rustdoc -\/-input-format json}.

rustdoc also supports personalizing the output from crates'
documentation, similar to markdown options.

\begin{itemize}
\itemsep1pt\parskip0pt\parsep0pt
\item
  \texttt{-\/-html-in-header FILE}: includes the contents of
  \texttt{FILE} at the end of the
  \texttt{\textless{}head\textgreater{}...\textless{}/head\textgreater{}}
  section.
\item
  \texttt{-\/-html-before-content FILE}: includes the contents of
  \texttt{FILE} directly after \texttt{\textless{}body\textgreater{}},
  before the rendered content (including the search bar).
\item
  \texttt{-\/-html-after-content FILE}: includes the contents of
  \texttt{FILE} after all the rendered content.
\end{itemize}

\section{Using the Documentation}\label{using-the-documentation}

The web pages generated by rustdoc present the same logical hierarchy
that one writes a library with. Every kind of item (function, struct,
etc) has its own color, and one can always click on a colored type to
jump to its documentation. There is a search bar at the top, which is
powered by some JavaScript and a statically-generated search index. No
special web server is required for the search.

\section{Testing the Documentation}\label{testing-the-documentation}

\texttt{rustdoc} has support for testing code examples which appear in
the documentation. This is helpful for keeping code examples up to date
with the source code.

To test documentation, the \texttt{-\/-test} argument is passed to
rustdoc:

\begin{verbatim}
rustdoc --test crate.rs
\end{verbatim}

\subsection{Defining tests}\label{defining-tests}

Rust documentation currently uses the markdown format, and rustdoc
treats all code blocks as testable-by-default unless they carry a
language tag of another language. In order to not run a test over a
block of code, the \texttt{ignore} string can be added to the
three-backtick form of markdown code block.

\begin{verbatim}
```
// This is a testable code block
```

```rust{.example}
// This is rust and also testable
```

```ignore
// This is not a testable code block
```

    // This is a testable code block (4-space indent)

```sh
# this is shell code and not tested
```
\end{verbatim}

You can specify that the test's execution should fail with the
\texttt{should\_fail} directive.

\begin{verbatim}
```should_fail
// This code block is expected to generate a failure when run
```
\end{verbatim}

You can specify that the code block should be compiled but not run with
the \texttt{no\_run} directive.

\begin{verbatim}
```no_run
// This code will be compiled but not executed
```
\end{verbatim}

Lastly, you can specify that a code block be compiled as if
\texttt{-\/-test} were passed to the compiler using the
\texttt{test\_harness} directive.

\begin{verbatim}
```test_harness
#[test]
fn foo() {
    fail!("oops! (will run & register as failure)")
}
```
\end{verbatim}

Rustdoc also supplies some extra sugar for helping with some tedious
documentation examples. If a line is prefixed with \texttt{\#}, then the
line will not show up in the HTML documentation, but it will be used
when testing the code block (NB. the space after the \texttt{\#} is
required, so that one can still write things like
\texttt{\#{[}deriving(Eq){]}}).

\begin{verbatim}
```
# /!\ The three following lines are comments, which are usually stripped off by
# the doc-generating tool.  In order to display them anyway in this particular
# case, the character following the leading '#' is not a usual space like in
# these first five lines but a non breakable one.
# // showing 'fib' in this documentation would just be tedious and detracts from
# // what's actually being documented.
# fn fib(n: int) { n + 2 }

spawn(proc() { fib(200); })
```
\end{verbatim}

The documentation online would look like
\texttt{spawn(proc() \{ fib(200); \})}, but when testing this code, the
\texttt{fib} function will be included (so it can compile).

\subsection{Running tests (advanced)}\label{running-tests-advanced}

Running tests often requires some special configuration to filter tests,
find libraries, or try running ignored examples. The testing framework
that rustdoc uses is build on crate \texttt{test}, which is also used
when you compile crates with rustc's \texttt{-\/-test} flag. Extra
arguments can be passed to rustdoc's test harness with the
\texttt{-\/-test-args} flag.

\begin{verbatim}
# Only run tests containing 'foo' in their name
$ rustdoc --test lib.rs --test-args 'foo'

# See what's possible when running tests
$ rustdoc --test lib.rs --test-args '--help'
\end{verbatim}

When testing a library, code examples will often show how functions are
used, and this code often requires \texttt{use}-ing paths from the
crate. To accommodate this, rustdoc will implicitly add
\texttt{extern crate \textless{}crate\textgreater{};} where
\texttt{\textless{}crate\textgreater{}} is the name of the crate being
tested to the top of each code example. This means that rustdoc must be
able to find a compiled version of the library crate being tested. Extra
search paths may be added via the \texttt{-L} flag to \texttt{rustdoc}.

\section{Standalone Markdown files}\label{standalone-markdown-files}

As well as Rust crates, rustdoc supports rendering pure Markdown files
into HTML and testing the code snippets from them. A Markdown file is
detected by a \texttt{.md} or \texttt{.markdown} extension.

There are 4 options to modify the output that Rustdoc creates.

\begin{itemize}
\itemsep1pt\parskip0pt\parsep0pt
\item
  \texttt{-\/-markdown-css PATH}: adds a
  \texttt{\textless{}link rel="stylesheet"\textgreater{}} tag pointing
  to \texttt{PATH}.
\item
  \texttt{-\/-html-in-header FILE}: includes the contents of
  \texttt{FILE} at the end of the
  \texttt{\textless{}head\textgreater{}...\textless{}/head\textgreater{}}
  section.
\item
  \texttt{-\/-html-before-content FILE}: includes the contents of
  \texttt{FILE} directly after \texttt{\textless{}body\textgreater{}},
  before the rendered content (including the title).
\item
  \texttt{-\/-html-after-content FILE}: includes the contents of
  \texttt{FILE} directly before \texttt{\textless{}/body\textgreater{}},
  after all the rendered content.
\end{itemize}

All of these can be specified multiple times, and they are output in the
order in which they are specified. The first line of the file being
rendered must be the title, prefixed with \texttt{\%} (e.g.~this page
has \texttt{\% Rust Documentation} on the first line).

Like with a Rust crate, the \texttt{-\/-test} argument will run the code
examples to check they compile, and obeys any \texttt{-\/-test-args}
flags. The tests are named after the last \texttt{\#} heading.

\end{document}
