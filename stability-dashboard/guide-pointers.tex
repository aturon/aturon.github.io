\documentclass[]{article}
\usepackage{lmodern}
\usepackage{amssymb,amsmath}
\usepackage{ifxetex,ifluatex}
\usepackage{fixltx2e} % provides \textsubscript
\ifnum 0\ifxetex 1\fi\ifluatex 1\fi=0 % if pdftex
  \usepackage[T1]{fontenc}
  \usepackage[utf8]{inputenc}
\else % if luatex or xelatex
  \ifxetex
    \usepackage{mathspec}
    \usepackage{xltxtra,xunicode}
  \else
    \usepackage{fontspec}
  \fi
  \defaultfontfeatures{Mapping=tex-text,Scale=MatchLowercase}
  \newcommand{\euro}{€}
\fi
% use upquote if available, for straight quotes in verbatim environments
\IfFileExists{upquote.sty}{\usepackage{upquote}}{}
% use microtype if available
\IfFileExists{microtype.sty}{\usepackage{microtype}}{}
\usepackage{longtable,booktabs}
\ifxetex
  \usepackage[setpagesize=false, % page size defined by xetex
              unicode=false, % unicode breaks when used with xetex
              xetex]{hyperref}
\else
  \usepackage[unicode=true]{hyperref}
\fi
\hypersetup{breaklinks=true,
            bookmarks=true,
            pdfauthor={},
            pdftitle={The Rust Pointer Guide},
            colorlinks=true,
            citecolor=blue,
            urlcolor=blue,
            linkcolor=magenta,
            pdfborder={0 0 0}}
\urlstyle{same}  % don't use monospace font for urls
\setlength{\parindent}{0pt}
\setlength{\parskip}{6pt plus 2pt minus 1pt}
\setlength{\emergencystretch}{3em}  % prevent overfull lines
\setcounter{secnumdepth}{5}

\title{The Rust Pointer Guide}

\begin{document}
\maketitle

0.12.0-pre (2c50add48 2014-07-27 23:03:52 -0700)

Copyright © 2011-2014 The Rust Project Developers. Licensed under the
\href{http://www.apache.org/licenses/LICENSE-2.0}{Apache License,
Version 2.0} or the \href{http://opensource.org/licenses/MIT}{MIT
license}, at your option.

This file may not be copied, modified, or distributed except according
to those terms.

{
\hypersetup{linkcolor=black}
\setcounter{tocdepth}{3}
\tableofcontents
}
Rust's pointers are one of its more unique and compelling features.
Pointers are also one of the more confusing topics for newcomers to
Rust. They can also be confusing for people coming from other languages
that support pointers, such as C++. This guide will help you understand
this important topic.

Be sceptical of non-reference pointers in Rust: use them for a
deliberate purpose, not just to make the compiler happy. Each pointer
type comes with an explanation about when they are appropriate to use.
Default to references unless you're in one of those specific situations.

You may be interested in the \hyperref[cheat-sheet]{cheat sheet}, which
gives a quick overview of the types, names, and purpose of the various
pointers.

\section{An introduction}\label{an-introduction}

If you aren't familiar with the concept of pointers, here's a short
introduction. Pointers are a very fundamental concept in systems
programming languages, so it's important to understand them.

\subsection{Pointer Basics}\label{pointer-basics}

When you create a new variable binding, you're giving a name to a value
that's stored at a particular location on the stack. (If you're not
familiar with the ``heap'' vs. ``stack'', please check out
\href{http://stackoverflow.com/questions/79923/what-and-where-are-the-stack-and-heap}{this
Stack Overflow question}, as the rest of this guide assumes you know the
difference.) Like this:

\texttt{\{rust\} let x = 5i; let y = 8i;} \textbar{} location \textbar{}
value \textbar{} \textbar{}----------\textbar{}-------\textbar{}
\textbar{} 0xd3e030 \textbar{} 5 \textbar{} \textbar{} 0xd3e028
\textbar{} 8 \textbar{}

We're making up memory locations here, they're just sample values.
Anyway, the point is that \texttt{x}, the name we're using for our
variable, corresponds to the memory location \texttt{0xd3e030}, and the
value at that location is \texttt{5}. When we refer to \texttt{x}, we
get the corresponding value. Hence, \texttt{x} is \texttt{5}.

Let's introduce a pointer. In some languages, there is just one type of
`pointer,' but in Rust, we have many types. In this case, we'll use a
Rust \textbf{reference}, which is the simplest kind of pointer.

\texttt{\{rust\} let x = 5i; let y = 8i; let z = \&y;}
\textbar{}location \textbar{} value \textbar{} \textbar{}--------
\textbar{}----------\textbar{} \textbar{}0xd3e030 \textbar{} 5
\textbar{} \textbar{}0xd3e028 \textbar{} 8 \textbar{} \textbar{}0xd3e020
\textbar{} 0xd3e028 \textbar{}

See the difference? Rather than contain a value, the value of a pointer
is a location in memory. In this case, the location of \texttt{y}.
\texttt{x} and \texttt{y} have the type \texttt{int}, but \texttt{z} has
the type \texttt{\&int}. We can print this location using the
\texttt{\{:p\}} format string:

```\{rust\} let x = 5i; let y = 8i; let z = \&y;

println!(``\{:p\}'', z); ```

This would print \texttt{0xd3e028}, with our fictional memory addresses.

Because \texttt{int} and \texttt{\&int} are different types, we can't,
for example, add them together:

```\{rust,ignore\} let x = 5i; let y = 8i; let z = \&y;

println!(``\{\}'', x + z); ```

This gives us an error:

\texttt{\{notrust,ignore\} hello.rs:6:24: 6:25 error: mismatched types: expected `int` but found `\&int` (expected int but found \&-ptr) hello.rs:6     println!("\{\}", x + z);                                   \^{}}

We can \textbf{dereference} the pointer by using the \texttt{*}
operator. Dereferencing a pointer means accessing the value at the
location stored in the pointer. This will work:

```\{rust\} let x = 5i; let y = 8i; let z = \&y;

println!(``\{\}'', x + *z); ```

It prints \texttt{13}.

That's it! That's all pointers are: they point to some memory location.
Not much else to them. Now that we've discussed the `what' of pointers,
let's talk about the `why.'

\subsection{Pointer uses}\label{pointer-uses}

Rust's pointers are quite useful, but in different ways than in other
systems languages. We'll talk about best practices for Rust pointers
later in the guide, but here are some ways that pointers are useful in
other languages:

In C, strings are a pointer to a list of \texttt{char}s, ending with a
null byte. The only way to use strings is to get quite familiar with
pointers.

Pointers are useful to point to memory locations that are not on the
stack. For example, our example used two stack variables, so we were
able to give them names. But if we allocated some heap memory, we
wouldn't have that name available. In C, \texttt{malloc} is used to
allocate heap memory, and it returns a pointer.

As a more general variant of the previous two points, any time you have
a structure that can change in size, you need a pointer. You can't tell
at compile time how much memory to allocate, so you've gotta use a
pointer to point at the memory where it will be allocated, and deal with
it at run time.

Pointers are useful in languages that are pass-by-value, rather than
pass-by-reference. Basically, languages can make two choices (this is
made up syntax, it's not Rust):

```\{notrust,ignore\} fn foo(x) \{ x = 5 \}

fn main() \{ i = 1 foo(i) // what is the value of i here? \} ```

In languages that are pass-by-value, \texttt{foo} will get a copy of
\texttt{i}, and so the original version of \texttt{i} is not modified.
At the comment, \texttt{i} will still be \texttt{1}. In a language that
is pass-by-reference, \texttt{foo} will get a reference to \texttt{i},
and therefore, can change its value. At the comment, \texttt{i} will be
\texttt{5}.

So what do pointers have to do with this? Well, since pointers point to
a location in memory\ldots{}

```\{notrust,ignore\} fn foo(\&int x) \{ *x = 5 \}

fn main() \{ i = 1 foo(\&i) // what is the value of i here? \} ```

Even in a language which is pass by value, \texttt{i} will be \texttt{5}
at the comment. You see, because the argument \texttt{x} is a pointer,
we do send a copy over to \texttt{foo}, but because it points at a
memory location, which we then assign to, the original value is still
changed. This pattern is called `pass-reference-by-value.' Tricky!

\subsection{Common pointer problems}\label{common-pointer-problems}

We've talked about pointers, and we've sung their praises. So what's the
downside? Well, Rust attempts to mitigate each of these kinds of
problems, but here are problems with pointers in other languages:

Uninitialized pointers can cause a problem. For example, what does this
program do?

\texttt{\{notrust,ignore\} \&int x; *x = 5; // whoops!}

Who knows? We just declare a pointer, but don't point it at anything,
and then set the memory location that it points at to be \texttt{5}. But
which location? Nobody knows. This might be harmless, and it might be
catastrophic.

When you combine pointers and functions, it's easy to accidentally
invalidate the memory the pointer is pointing to. For example:

```\{notrust,ignore\} fn make\_pointer(): \&int \{ x = 5;

\begin{verbatim}
return &x;
\end{verbatim}

\}

fn main() \{ \&int i = make\_pointer(); *i = 5; // uh oh! \} ```

\texttt{x} is local to the \texttt{make\_pointer} function, and
therefore, is invalid as soon as \texttt{make\_pointer} returns. But we
return a pointer to its memory location, and so back in \texttt{main},
we try to use that pointer, and it's a very similar situation to our
first one. Setting invalid memory locations is bad.

As one last example of a big problem with pointers, \textbf{aliasing}
can be an issue. Two pointers are said to alias when they point at the
same location in memory. Like this:

```\{notrust,ignore\} fn mutate(\&int i, int j) \{ *i = j; \}

fn main() \{ x = 5; y = \&x; z = \&x; //y and z are aliased

run\_in\_new\_thread(mutate, y, 1); run\_in\_new\_thread(mutate, z,
100);

// what is the value of x here? \} ```

In this made-up example, \texttt{run\_in\_new\_thread} spins up a new
thread, and calls the given function name with its arguments. Since we
have two threads, and they're both operating on aliases to \texttt{x},
we can't tell which one finishes first, and therefore, the value of
\texttt{x} is actually non-deterministic. Worse, what if one of them had
invalidated the memory location they pointed to? We'd have the same
problem as before, where we'd be setting an invalid location.

\subsection{Conclusion}\label{conclusion}

That's a basic overview of pointers as a general concept. As we alluded
to before, Rust has different kinds of pointers, rather than just one,
and mitigates all of the problems that we talked about, too. This does
mean that Rust pointers are slightly more complicated than in other
languages, but it's worth it to not have the problems that simple
pointers have.

\section{References}\label{references}

The most basic type of pointer that Rust has is called a `reference.'
Rust references look like this:

```\{rust\} let x = 5i; let y = \&x;

println!(``\{\}'', *y); println!(``\{:p\}'', y); println!(``\{\}'', y);
```

We'd say ``\texttt{y} is a reference to \texttt{x}.'' The first
\texttt{println!} prints out the value of \texttt{y}'s referent by using
the dereference operator, \texttt{*}. The second one prints out the
memory location that \texttt{y} points to, by using the pointer format
string. The third \texttt{println!} \emph{also} prints out the value of
\texttt{y}'s referent, because \texttt{println!} will automatically
dereference it for us.

Here's a function that takes a reference:

\texttt{\{rust\} fn succ(x: \&int) -\textgreater{} int \{ *x + 1 \}}

You can also use \texttt{\&} as an operator to create a reference, so we
can call this function in two different ways:

```\{rust\} fn succ(x: \&int) -\textgreater{} int \{ *x + 1 \}

fn main() \{

\begin{verbatim}
let x = 5i;
let y = &x;

println!("{}", succ(y));
println!("{}", succ(&x));
\end{verbatim}

\} ```

Both of these \texttt{println!}s will print out \texttt{6}.

Of course, if this were real code, we wouldn't bother with the
reference, and just write:

\texttt{\{rust\} fn succ(x: int) -\textgreater{} int \{ x + 1 \}}

References are immutable by default:

```\{rust,ignore\} let x = 5i; let y = \&x;

*y = 5; // error: cannot assign to immutable dereference of
\texttt{\&}-pointer \texttt{*y} ```

They can be made mutable with \texttt{mut}, but only if its referent is
also mutable. This works:

\texttt{\{rust\} let mut x = 5i; let y = \&mut x;}

This does not:

\texttt{\{rust,ignore\} let x = 5i; let y = \&mut x; // error: cannot borrow immutable local variable `x` as mutable}

Immutable pointers are allowed to alias:

\texttt{\{rust\} let x = 5i; let y = \&x; let z = \&x;}

Mutable ones, however, are not:

\texttt{\{rust,ignore\} let x = 5i; let y = \&mut x; let z = \&mut x; // error: cannot borrow `x` as mutable more than once at a time}

Despite their complete safety, a reference's representation at runtime
is the same as that of an ordinary pointer in a C program. They
introduce zero overhead. The compiler does all safety checks at compile
time. The theory that allows for this was originally called
\textbf{region pointers}. Region pointers evolved into what we know
today as \textbf{lifetimes}.

Here's the simple explanation: would you expect this code to compile?

\texttt{\{rust,ignore\} fn main() \{     println!("\{\}", x);     let x = 5; \}}

Probably not. That's because you know that the name \texttt{x} is valid
from where it's declared to when it goes out of scope. In this case,
that's the end of the \texttt{main} function. So you know this code will
cause an error. We call this duration a `lifetime'. Let's try a more
complex example:

```\{rust\} fn main() \{ let x = \&mut 5i;

\begin{verbatim}
if *x < 10 {
    let y = &x;

    println!("Oh no: {}", y);
    return;
}

*x -= 1;

println!("Oh no: {}", x);
\end{verbatim}

\} ```

Here, we're borrowing a pointer to \texttt{x} inside of the \texttt{if}.
The compiler, however, is able to determine that that pointer will go
out of scope without \texttt{x} being mutated, and therefore, lets us
pass. This wouldn't work:

```\{rust,ignore\} fn main() \{ let x = \&mut 5i;

\begin{verbatim}
if *x < 10 {
    let y = &x;
    *x -= 1;

    println!("Oh no: {}", y);
    return;
}

*x -= 1;

println!("Oh no: {}", x);
\end{verbatim}

\} ```

It gives this error:

\texttt{\{notrust,ignore\} test.rs:5:8: 5:10 error: cannot assign to `*x` because it is borrowed test.rs:5         *x -= 1;                   \^{}\textasciitilde{} test.rs:4:16: 4:18 note: borrow of `*x` occurs here test.rs:4         let y = \&x;                           \^{}\textasciitilde{}}

As you might guess, this kind of analysis is complex for a human, and
therefore hard for a computer, too! There is an entire
\href{guide-lifetimes.html}{guide devoted to references and lifetimes}
that goes into lifetimes in great detail, so if you want the full
details, check that out.

\subsection{Best practices}\label{best-practices}

In general, prefer stack allocation over heap allocation. Using
references to stack allocated information is preferred whenever
possible. Therefore, references are the default pointer type you should
use, unless you have specific reason to use a different type. The other
types of pointers cover when they're appropriate to use in their own
best practices sections.

Use references when you want to use a pointer, but do not want to take
ownership. References just borrow ownership, which is more polite if you
don't need the ownership. In other words, prefer:

\texttt{\{rust\} fn succ(x: \&int) -\textgreater{} int \{ *x + 1 \}}

to

\texttt{\{rust\} fn succ(x: Box\textless{}int\textgreater{}) -\textgreater{} int \{ *x + 1 \}}

As a corollary to that rule, references allow you to accept a wide
variety of other pointers, and so are useful so that you don't have to
write a number of variants per pointer. In other words, prefer:

\texttt{\{rust\} fn succ(x: \&int) -\textgreater{} int \{ *x + 1 \}}

to

```\{rust\} fn box\_succ(x: Box) -\textgreater{} int \{ *x + 1 \}

fn rc\_succ(x: std::rc::Rc) -\textgreater{} int \{ *x + 1 \} ```

\section{Boxes}\label{boxes}

\texttt{Box\textless{}T\textgreater{}} is Rust's `boxed pointer' type.
Boxes provide the simplest form of heap allocation in Rust. Creating a
box looks like this:

\texttt{\{rust\} let x = box(std::boxed::HEAP) 5i;}

\texttt{box} is a keyword that does `placement new,' which we'll talk
about in a bit. \texttt{box} will be useful for creating a number of
heap-allocated types, but is not quite finished yet. In the meantime,
\texttt{box}'s type defaults to \texttt{std::boxed::HEAP}, and so you
can leave it off:

\texttt{\{rust\} let x = box 5i;}

As you might assume from the \texttt{HEAP}, boxes are heap allocated.
They are deallocated automatically by Rust when they go out of scope:

```\{rust\} \{ let x = box 5i;

\begin{verbatim}
// stuff happens
\end{verbatim}

\} // x is destructed and its memory is free'd here ```

However, boxes do \emph{not} use reference counting or garbage
collection. Boxes are what's called an \textbf{affine type}. This means
that the Rust compiler, at compile time, determines when the box comes
into and goes out of scope, and inserts the appropriate calls there.
Furthermore, boxes are a specific kind of affine type, known as a
\textbf{region}. You can read more about regions
\href{http://www.cs.umd.edu/projects/cyclone/papers/cyclone-regions.pdf}{in
this paper on the Cyclone programming language}.

You don't need to fully grok the theory of affine types or regions to
grok boxes, though. As a rough approximation, you can treat this Rust
code:

```\{rust\} \{ let x = box 5i;

\begin{verbatim}
// stuff happens
\end{verbatim}

\} ```

As being similar to this C code:

```\{notrust,ignore\} \{ int \emph{x; x = (int })malloc(sizeof(int));

\begin{verbatim}
// stuff happens

free(x);
\end{verbatim}

\} ```

Of course, this is a 10,000 foot view. It leaves out destructors, for
example. But the general idea is correct: you get the semantics of
\texttt{malloc}/\texttt{free}, but with some improvements:

\begin{enumerate}
\def\labelenumi{\arabic{enumi}.}
\itemsep1pt\parskip0pt\parsep0pt
\item
  It's impossible to allocate the incorrect amount of memory, because
  Rust figures it out from the types.
\item
  You cannot forget to \texttt{free} memory you've allocated, because
  Rust does it for you.
\item
  Rust ensures that this \texttt{free} happens at the right time, when
  it is truly not used. Use-after-free is not possible.
\item
  Rust enforces that no other writeable pointers alias to this heap
  memory, which means writing to an invalid pointer is not possible.
\end{enumerate}

See the section on references or the
\href{guide-lifetimes.html}{lifetimes guide} for more detail on how
lifetimes work.

Using boxes and references together is very common. For example:

```\{rust\} fn add\_one(x: \&int) -\textgreater{} int \{ *x + 1 \}

fn main() \{ let x = box 5i;

\begin{verbatim}
println!("{}", add_one(&*x));
\end{verbatim}

\} ```

In this case, Rust knows that \texttt{x} is being `borrowed' by the
\texttt{add\_one()} function, and since it's only reading the value,
allows it.

We can borrow \texttt{x} multiple times, as long as it's not
simultaneous:

```\{rust\} fn add\_one(x: \&int) -\textgreater{} int \{ *x + 1 \}

fn main() \{ let x = box 5i;

\begin{verbatim}
println!("{}", add_one(&*x));
println!("{}", add_one(&*x));
println!("{}", add_one(&*x));
\end{verbatim}

\} ```

Or as long as it's not a mutable borrow. This will error:

```\{rust,ignore\} fn add\_one(x: \&mut int) -\textgreater{} int \{ *x +
1 \}

fn main() \{ let x = box 5i;

\begin{verbatim}
println!("{}", add_one(&*x)); // error: cannot borrow immutable dereference 
                              // of `&`-pointer as mutable
\end{verbatim}

\} ```

Notice we changed the signature of \texttt{add\_one()} to request a
mutable reference.

\section{Best practices}\label{best-practices-1}

Boxes are appropriate to use in two situations: Recursive data
structures, and occasionally, when returning data.

\subsection{Recursive data structures}\label{recursive-data-structures}

Sometimes, you need a recursive data structure. The simplest is known as
a `cons list':

```\{rust\} \#{[}deriving(Show){]} enum List \{ Cons(T,
Box\textgreater{}), Nil, \}

fn main() \{ let list: List = Cons(1, box Cons(2, box Cons(3, box
Nil))); println!(``\{\}'', list); \} ```

This prints:

\texttt{\{notrust,ignore\} Cons(1, box Cons(2, box Cons(3, box Nil)))}

The reference to another \texttt{List} inside of the \texttt{Cons} enum
variant must be a box, because we don't know the length of the list.
Because we don't know the length, we don't know the size, and therefore,
we need to heap allocate our list.

Working with recursive or other unknown-sized data structures is the
primary use-case for boxes.

\subsection{Returning data}\label{returning-data}

This is important enough to have its own section entirely. The TL;DR is
this: you don't generally want to return pointers, even when you might
in a language like C or C++.

See \hyperref[returning-pointers]{Returning Pointers} below for more.

\section{Rc and Arc}\label{rc-and-arc}

This part is coming soon.

\subsection{Best practices}\label{best-practices-2}

This part is coming soon.

\section{Gc}\label{gc}

The \texttt{Gc\textless{}T\textgreater{}} type exists for historical
reasons, and is
\href{https://github.com/rust-lang/rust/issues/7929}{still used
internally} by the compiler. It is not even a `real' garbage collected
type at the moment.

In the future, Rust may have a real garbage collected type, and so it
has not yet been removed for that reason.

\subsection{Best practices}\label{best-practices-3}

There is currently no legitimate use case for the
\texttt{Gc\textless{}T\textgreater{}} type.

\section{Raw Pointers}\label{raw-pointers}

This part is coming soon.

\subsection{Best practices}\label{best-practices-4}

This part is coming soon.

\hyperdef{}{returning-pointers}{\section{Returning
Pointers}\label{returning-pointers}}

In many languages with pointers, you'd return a pointer from a function
so as to avoid a copying a large data structure. For example:

```\{rust\} struct BigStruct \{ one: int, two: int, // etc one\_hundred:
int, \}

fn foo(x: Box) -\textgreater{} Box \{ return box *x; \}

fn main() \{ let x = box BigStruct \{ one: 1, two: 2, one\_hundred: 100,
\};

\begin{verbatim}
let y = foo(x);
\end{verbatim}

\} ```

The idea is that by passing around a box, you're only copying a pointer,
rather than the hundred \texttt{int}s that make up the
\texttt{BigStruct}.

This is an antipattern in Rust. Instead, write this:

```\{rust\} struct BigStruct \{ one: int, two: int, // etc one\_hundred:
int, \}

fn foo(x: Box) -\textgreater{} BigStruct \{ return *x; \}

fn main() \{ let x = box BigStruct \{ one: 1, two: 2, one\_hundred: 100,
\};

\begin{verbatim}
let y = box foo(x);
\end{verbatim}

\} ```

This gives you flexibility without sacrificing performance.

You may think that this gives us terrible performance: return a value
and then immediately box it up ?! Isn't that the worst of both worlds?
Rust is smarter than that. There is no copy in this code. main allocates
enough room for the `box , passes a pointer to that memory into foo as
x, and then foo writes the value straight into that pointer. This writes
the return value directly into the allocated box.

This is important enough that it bears repeating: pointers are not for
optimizing returning values from your code. Allow the caller to choose
how they want to use your output.

\section{Creating your own Pointers}\label{creating-your-own-pointers}

This part is coming soon.

\subsection{Best practices}\label{best-practices-5}

This part is coming soon.

\hyperdef{}{cheat-sheet}{\section{Cheat Sheet}\label{cheat-sheet}}

Here's a quick rundown of Rust's pointer types:

\begin{longtable}[c]{@{}lll@{}}
\toprule\addlinespace
Type & Name & Summary
\\\addlinespace
\midrule\endhead
\texttt{\&T} & Reference & Allows one or more references to read
\texttt{T}
\\\addlinespace
\texttt{\&mut T} & Mutable Reference & Allows a single reference to
\\\addlinespace
& & read and write \texttt{T}
\\\addlinespace
\texttt{Box\textless{}T\textgreater{}} & Box & Heap allocated \texttt{T}
with a single owner
\\\addlinespace
& & that may read and write \texttt{T}.
\\\addlinespace
\texttt{Rc\textless{}T\textgreater{}} & ``arr cee'' pointer & Heap
allocated \texttt{T} with many readers
\\\addlinespace
\texttt{Arc\textless{}T\textgreater{}} & Arc pointer & Same as above,
but safe sharing across
\\\addlinespace
& & threads
\\\addlinespace
\texttt{*const T} & Raw pointer & Unsafe read access to \texttt{T}
\\\addlinespace
\texttt{*mut T} & Mutable raw pointer & Unsafe read and write access to
\texttt{T}
\\\addlinespace
\bottomrule
\end{longtable}

\section{Related resources}\label{related-resources}

\begin{itemize}
\itemsep1pt\parskip0pt\parsep0pt
\item
  \href{std/boxed/index.html}{API documentation for Box}
\item
  \href{guide-lifetimes.html}{Lifetimes guide}
\item
  \href{http://www.cs.umd.edu/projects/cyclone/papers/cyclone-regions.pdf}{Cyclone
  paper on regions}, which inspired Rust's lifetime system
\end{itemize}

\end{document}
